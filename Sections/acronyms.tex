% !TEX root = MainDocument.tex
\section*{Abkürzungsverzeichnis}

% Appear in Table of contents
\addcontentsline{toc}{section}{Abkürzungsverzeichnis}

\begin{acronym}[HRTEM]

\acro{adc} [ADC] {Analog- Digital- Wandler, engl. \foreignlanguage{english}{analog-digital-converter}, elektronisches Bauteil zum quantisieren analoger in digitale Werte}

\acro{osr} [OSR] {Überabtast-
verhältnis, engl. \foreignlanguage{english}{oversampling ratio}, Verhältnis von Modulatorfrequenz ind Ausgangsdatenrate}

\acro{sps} [SPS] {Abtastungen in einer Sekunde, engl. \foreignlanguage{english}{samples per second}, beschreibt die Datenrate eines AD- Wandlers}

\acro{pga} [PGA] {Verstärker mit programmierbarer Verstärkung, engl. \foreignlanguage{english}{ Programmable Gain Amplifier},  Operationsverstärker, der als nicht invertierter Verstärker beschaltet ist und einen programmierbaren Analogmultiplexer verbaut hat}

\acro{bocs} [BOCS] { Burn-Out-Stromquellen, engl. \foreignlanguage{english}{Burn Out Current Source}, Stromquellen für den Sensor Test}

\acro{spi} [SPI] {engl. \foreignlanguage{english}{Serial Peripheral Interface},  Bus-System „lockeren“ Standards für synchrone serielle Datenbusse (Synchronous Serial Ports) nach dem Master-Slave-Prinzip}

\acro{lsb} [LSB] { Bit mit der geringstejn Bitwertigkeit, engl. \foreignlanguage{english}{Least Significant Bit}}

\acro{rtd} [RTD] { Messung der Temperatur mit einem Widerstand, engl. \foreignlanguage{english}{Resistance Temperature Detector}}

\acro{cpu} [CPU] { zentrale Recheneinheit, engl. \foreignlanguage{english}{Central Processing Unit}, eines Rechners oder Mikrocontrollers. Umgangssprachlich oft als Prozessor bezeichnet.}

\acro{dma} [DMA] { Direkter Speicherzugriff, engl. \foreignlanguage{english}{Direct Memory Access}, seperater Speicherbus mit Controller um Daten ohne Umwege über die CPU direkt in den Arbeitsspeicher schreiben zu können.}

\acro{mcu} [MCU] { Mikrocontroller, engl. \foreignlanguage{english}{Micor Controlling Unit}, Einplatinenchips, die gleichzeitig einen Prozessor und Peripheriefunktionen enthalten}


\end{acronym}